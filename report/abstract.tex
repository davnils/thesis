\newenvironment{abstractpage}
  {\cleardoublepage\vspace*{\fill}\thispagestyle{empty}}
  {\vfill\cleardoublepage}
\newenvironment{polyAbstract}[1]
  {\bigskip\selectlanguage{#1}%
   \begin{center}\bfseries\abstractname\end{center}}
  {\par\bigskip}

\begin{abstractpage}
\begin{polyAbstract}{english}
This master's thesis concerns three different areas in the field of fault detection in photovoltaic systems.
Previous studies have concerned homogeneous systems with a large set of parameters being observed, while this study is focused on a more restrictive case.
The first problem is to discover immediate faults occuring in solar panels.
A new online algorithm is developed based on similarity measures within a single installation.
It performs reliably and is able to detect all significant faults over a certain threshold.
The second problem concerns measuring degradation over time.
A modifed approach is taken based on repetitive conditions, and performs well given certain assumptions.
Finally the third problem is to differentiate solar panel faults from partial shading.
Here a clustering algorithm DBSCAN is applied on data in order to locate clusters of faults in the solar plane, demonstrating good performance in certain situations.
It also demonstrates issues with missclassification of real faults due to clustering.

\end{polyAbstract}

\begin{polyAbstract}{swedish}
Det här är en uppsats på masternivå om feldetektering av fotovoltaiska system.
Tidigare studier har fokuserat på homogena system med fler observerade parametrar, vilket generaliseras i den här studien.
Den första delen är feldetektering av snabba felförlopp i solpaneler.
En ny algoritm presenteras baserad på grader av likhet inom en enskild solpanelsinstallation.
Den presterar väl och är kapabel att hitta alla signifikanta fel över en viss nivå.
Den andra delen består av att mäta degradering av solpaneler över tid.
En variant av tidigare resultat presenteras som baseras på återkommande förhållanden, vilken presterar väl givet vissa antaganden.
Slutligen hanteras detektering av partiell skuggning och urskiljning av detta från riktiga panelfel.
Lösningen är en algoritm DBSCAN som hittar kluster av data i solplanet och den påvisar god prestanda i vissa situationer.
Det finns även situationer då riktiga fel missklassificeras p.g.a att de är klustrade.

\end{polyAbstract}
\end{abstractpage}

\selectlanguage{english}
