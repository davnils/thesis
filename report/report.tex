\documentclass[a4paper,11pt]{report}
\pdfoutput=1

\usepackage{algpseudocode}
\usepackage{algorithm}
\usepackage[english,swedish]{babel}
\usepackage[T1]{fontenc}
\usepackage[utf8x]{inputenc}
\usepackage{listings, babel}
\usepackage{graphicx}
\usepackage[colorlinks=true,linktoc=page]{hyperref}
\usepackage[nonumberlist]{glossaries}
\usepackage{subcaption}
\lstset{breaklines=true,basicstyle=\ttfamily}
\usepackage[margin=2cm]{geometry}

\selectlanguage{english}

\newenvironment{abstractpage}
  {\cleardoublepage\vspace*{\fill}\thispagestyle{empty}}
  {\vfill\cleardoublepage}
\newenvironment{polyAbstract}[1]
  {\bigskip\selectlanguage{#1}%
   \begin{center}\bfseries\abstractname\end{center}}
  {\par\bigskip}

\usepackage{lipsum}

\title{Fault detection in photovoltaic systems}
\author{David Nilsson, davnils@kth.se}

\newglossaryentry{MPP}
{
  name=MPP,
  description={maximum power point corresponding to an optimal load}
}

\newglossaryentry{MPPT}
{
  name=MPPT,
  description={maximum power point tracking}
}

\newglossaryentry{solar-module}
{
  name=solar module,
  description={An enclosed box containing several interconnected solar cells}
}

\newglossaryentry{iv-curve}
{
  name=I-V curve,
  description={Plot of panel current as a function of panel voltage}
}


\makeglossaries
\glsaddall


\begin{document}
\pagenumbering{gobble}

\maketitle

\begin{abstractpage}
\begin{polyAbstract}{english}
    \lipsum[1]
\end{polyAbstract}

%TODO: Include translated version of the thesis title
\begin{polyAbstract}{swedish}
    \lipsum[1]
\end{polyAbstract}
\end{abstractpage}

\selectlanguage{english}

\section*{Preface}
TBD

\tableofcontents

\printglossaries
\clearpage

\pagenumbering{arabic}

\chapter{Introduction}
TBD

\section{Overview of fault detection}
TBD

\section{Intended readers}
The main target audience interested in this thesis are companies building products in the field of PV power electronics.
Fault detection in solar panels is an active research area and continually evolves, demonstrating an academic interest as well.
The main party interested in the results is of course the supervising company, whom are likely to
integrate a functional solution into their final product.
While being a fairly specialized thesis, it should be comprehensible to anyone with a basic background in statistics.

\section{Scope of this thesis}
The most important scope limitation is that systems should only be studied passively, i.e. the available measurements are used to detect faulty panels.  
Some technical constraints are present as well:
\begin{itemize}
\item The solar panels have 60 or 72 cells built of mono- or poly-crystalline silicon

\item The installations have 14 to 24 panels connected within a small geographic area

\item The available measurements (from each panel) are:
$U$ \footnote{Panel voltage at an optimal load},
$I$ \footnote{Panel current at an optimal load} and
$T_{module}$ \footnote{Low resolution temperature measurement from within the PV inverter}.

\end{itemize}

This reflects the consumer market of smaller panel installations with some of the most common panel types.
The type of solar panels constrains the possible output power range and defines appropriate parameters for testing.
Due to time constraints, the simulation of solar panels will largely be based on existing data streams.
These will be used a basis of generating approximate individual panel curves, limiting the amount of effort required.

The supervising provide data feeds providing real time measurements of several PV installations in Sweden.
This data is accumulated in a central database and can be analyzed.
This data has been used in order to verify real-life performance of the classification system, primarily to verify that working systems are not classified as faulty.

Additionally there are other sources that can be used to build realistic simulation models, such as measured solar energy over several years.

\section{Ethical considerations}
TBD


\chapter{Background}
TBD

\section{Theory of a solar cell}
TBD

\section{Systems of solar panels}
TBD

\section{Solar power in practice}
TBD


\chapter{Theory of fault detection}
TBD

\section{Issues in photovoltaic systems}
TBD

\section{Partial shading of of solar panels}
TBD

\section{Approaches to fault detection}
TBD

\section{Measuring degradation}
TBD

\begin{thebibliography}{10}

\bibitem{BHKK13}
A.~{Bj{\"o}rklund}, P.~Kaski, and {\L}.~Kowalik.
\newblock Counting thin subgraphs via packings faster than meet-in-the-middle time.
\newblock {\em arXiv} preprint arXiv:1306.4111, 2013.

\end{thebibliography}

\end{document}
